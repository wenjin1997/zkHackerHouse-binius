% \documentclass[11pt,twoside,a4paper]{article}
% \usepackage{times}

% \usepackage{xeCJK}

% \setmainfont{Times New Roman}

% \setCJKmainfont{Songti SC}
\documentclass[10pt]{ctexart}
% \usepackage[UTF-8]{ctex}
\usepackage{amsmath}
\usepackage{amsthm} % 根据 amsthm 的手册, amsthm 的加载要在 amsmath 之后
\usepackage{amssymb}  %为了能使用\mathbb{H} 
\usepackage{booktabs}
\usepackage{multirow}
\usepackage{tabularx}
\usepackage{xcolor}
\usepackage[colorlinks,linkcolor=blue]{hyperref} % 使用超链接
\usepackage{pdfpages}
\usepackage{geometry}
\geometry{a4paper,scale=0.8}
\usepackage{graphicx} %插入图片的宏包
\usepackage{float} %设置图片浮动位置的宏包
\usepackage{subfigure} %插入多图时用子图显示的宏包
\usepackage{graphicx}
\usepackage{enumerate}
% \usepackage{natbib}
% \newcommand{\upcite}[1][{\setcitestyle{square,super}}\cite{#1}]

\usepackage{listings}

\lstset{
 columns=fixed,       
 numbers=left,                                        % 在左侧显示行号
 numberstyle=\tiny\color{gray},                       % 设定行号格式
 frame=none,                                          % 不显示背景边框
 backgroundcolor=\color[RGB]{245,245,244},            % 设定背景颜色
 keywordstyle=\color[RGB]{40,40,255},                 % 设定关键字颜色
 numberstyle=\footnotesize\color{darkgray},           
 commentstyle=\it\color[RGB]{0,96,96},                % 设置代码注释的格式
 stringstyle=\rmfamily\slshape\color[RGB]{128,0,0},   % 设置字符串格式
 showstringspaces=false,                              % 不显示字符串中的空格
%language=c++,                                        % 设置语言
}

\newtheorem{definition}{定义}
\newtheorem{lemma}{引理}
\newtheorem{theorem}{定理}
\newtheorem{example}{例}



\title{Notes of binius}
\author{Jade Xie}
\date{\today}
\begin{document}
\maketitle
\tableofcontents
\section{TO-DO}
\begin{itemize}
    \item 多项式$X^2 + X_{\iota - 2} \cdot X + 1$在$\mathcal{T}_{\iota - 1}[X_{\iota - 1}]$上是不可约的(见\cite{iterated-quadratic}Thm. 1).
    \item 两个大数相乘并作算法复杂度的分析.
\end{itemize}
\section{介绍}
% 对于例如Keccak-256哈希函数,在CPU上的运算比较快,但却是zkSnark的瓶颈.该问题源于计算机使用的基本数据类型与当今SNARK使用的数据类型是不匹配的。



% 在数学上,
\section{Background}
\subsection{Polynomials}

\subsection{Binary Towers}
\subsubsection{子域与扩张}
在具体深入binius论文\cite{binius} 2.3 Binary Towers 细节之前,先给出数学上关于域扩张的知识.

首先看看什么是扩域.下面参考《抽象代数》\cite{abstract-algerb}中的描述.设$E$是域,$F$是其\textbf{子域}(即$F \subset E$且$F$按照$E$中的运算成为域,二者乘法单位元同一),则称$E$是$F$的\textbf{扩张}或\textbf{扩域}(\textbf{extension field}),记为$E/F$.理解一下,意思是域$F$是域$E$的子域就构成一个扩域$E/F$,$F$是$E$的子域表示的意思是保持$F$中的两个二元运算,它们在两个域中是同样的两个二元运算,并且两个域的乘法单位元是同一个.扩域也可以用线性空间的角度来看,由$E$是$F$的扩域,特别可知$E$是$F$上的\textbf{线性空间}.

\begin{definition}
    如果一个代数系统$(V;+,\cdot ; \mathbb{P})$满足下列性质,那么就称为数域 $\mathbb{P}$上的一个\textbf{线性空间}.
    \begin{enumerate}[(1)]
        \item 向量加法的交换律:$\forall \alpha ,\beta \in V, \alpha + \beta = \beta + \alpha$.
        \item 向量加法的结合律:$\forall \alpha, \beta, \gamma \in V,(\alpha + \beta) + \gamma = \alpha + (\beta + \gamma)$.
        \item 向量加法有零元:$\exists \theta \in V, \forall \alpha \in V, \alpha + \theta = \alpha$.
        \item 向量加法有负元:$\forall \alpha \in V, \exists \alpha \prime \in V, \alpha + \alpha \prime = \theta$.
        \item 标量乘法对向量加法有分配律:$\forall \alpha, \beta \in V, \forall k \in \mathbb{P}, k \cdot (\alpha + \beta) = k \cdot \alpha + k \cdot \beta$.
        \item 标量乘法对域加法有分配律:$\forall \alpha \in V, \forall k, l \in \mathbb{P}, (k + l) \cdot \alpha = k \cdot \alpha + k \cdot \alpha$.
        \item 标量乘法与标量的域乘法相容:$\forall \alpha \in V, \forall k, l \in \mathbb{P}, (kl) \cdot \alpha = k \cdot (l \cdot \alpha)$.
        \item 标量乘法有单位元:$\forall \alpha \in V, 1 \cdot \alpha = \alpha$.
    \end{enumerate}
    满足以上八条性质便可称其为线性空间,在不引起混淆的情况下也可记为$V$.$k \cdot \alpha$也可沿用几何空间中向量数乘的习惯为$k \alpha$.
\end{definition}
根据线性空间的定义,这里$E/F$,$E$是$F$的线性空间,也就是说:
\begin{enumerate}[(1)]
    \item $E$是一个加法阿贝尔群.
    \item $F$中的元素与$E$中的元素之间有(数乘)运算且满足:对任意$c,c \prime \in F$,$\alpha$,$ \alpha \prime \in E$有
    \begin{enumerate}[(i)]
        \item $c \alpha \in E$
        \item $c (\alpha + \alpha \prime) = c \alpha + c \alpha \prime$
        \item $(c + c \prime) \alpha = c \alpha + c \prime \alpha$
        \item $(c c \prime) \alpha = c (c \prime \alpha)$
        \item $1 \cdot \alpha = \alpha$
    \end{enumerate}
\end{enumerate}

扩域$E$作为$F$上的线性空间,其维数称为扩张次数,记为$[E:F]$;此线性空间的基称为扩张$E/F$的基,或$E$的$F$-基.当$[E:F]$有限时,称$E/F$为有限扩张.而$[E:F]=1$意味着$E=F$.这里扩域$E$作为$F$上的线性空间,还可以这样来表述:设$\{\alpha_1, \alpha_2, \ldots, \alpha_n \}$是$E/F$的基,则对于$E$中的任意一个元素$v \in E$,都可以表示为
\begin{displaymath}
    v = \sum_{j}\omega_j \alpha_j = \omega_0 \alpha_0 + \omega_1 \alpha_1 + \cdots + \omega_n \alpha_n, \omega_j \in F, j = 0, 1, \ldots, n.
\end{displaymath}
意思就是$E$中的所有元素都可以用$F$中的元素通过$E$的$F$-基进行线性表出.不将基显式地写出来,也可以将$E$看作是$F$上的向量空间(vector space,其实向量空间就是线性空间\cite{vector-space}),例如上述的 $v$ 就能用一个向量来表示
\begin{displaymath}
    v = (\omega_0, \omega_1, \omega_2, \ldots, \omega_n)
\end{displaymath}

\begin{definition}
    \begin{enumerate}[(1)]
        \item 设$E/F$为域的扩张,$\alpha \in E$,称$\alpha$是$F$上的\textbf{代数元素}(algebraic element)是指:$\alpha$是$F$上的某多项式$f(x) \in F[x]$的根,即$f(\alpha) = 0$,也就是说存在正整数$n$和不全为$0$的$c_0$,$c_1$,$\cdots$,$c_n \in F$使得$c_n a^n + \cdots + c_1a + c_0 = 0$(称$f(x)$是$\alpha$的化零多项式).
        \item 如果$E$中所有元素都是$F$上的代数元素,则称$E/F$是\textbf{代数扩张}(algebraic extension).非代数元素为\textbf{超越}元素,非代数扩张为\textbf{超越扩张}(transcendent extension).
        \item 若复数$\alpha$是$\mathbb{Q}$上的代数元素,则称$\alpha$为代数数,否则称为超越数.
    \end{enumerate}
\end{definition}

在域扩张中,有很重要的一个定理,是域扩张的有力工具,定理如下.
\begin{theorem}[单代数扩张]\label{theorem: field extension}
    设$F$是任一域,$p(x) \in F[x]$是任一个$n(>1)$次不可约多项式,则存在$F$的$n$次单扩张$E = F(\alpha)$,且$\alpha$是$p(x)$的根.事实上,商环$E = F[x]/(p(x))$为域.视同构$F \simeq \bar{F}$为相等(对$b \in F$视为$b = \bar{b}$),则$E$是$F$的$n$次扩域,$\alpha = \bar{x}$ 是$p(X)$的根,且
    \begin{displaymath}
        E = F(\alpha) = \{b_0 + b_1 \alpha +  \cdots + b_{n - 1} \alpha^{n - 1} | b_0, b_1, \ldots , b_{n-1} \in F\},
    \end{displaymath}
    (这里$\overline{g(x)}$表示$g(x)$ 的模$(p(x))$同余类,$\bar{F} = \{\bar{b}:b \in F\}$).
\end{theorem}

\begin{example}[复数域的引入]
    设$F=\mathbb{R}$是实数域,$p(X) = X^2 + 1$不可约,$n=2$.则商环
    \begin{displaymath}
        E = \mathbb{R}[x]/(x^2+1)=\{ \bar{a}+\bar{b}\bar{x}|a,b\in \mathbb{R} \} = \{a + bi |a,b \in \mathbb{R}\}
    \end{displaymath}
    (其中$i = \bar{x}$,且对实数$b$记$\bar{b}=b$).于是$0=x^2+1=\bar{x}^2+1=i^2+1$,$i^2=-1$,故常记$i=\sqrt{-1}$,$E=\{a+bi\}$就是复数域$\mathbb{C}$.这是引入复数域的最严格途径.
\end{example}

论文\cite{binius}中的域扩张构造就是按照这个定理思路进行扩张的.先选定第一个域$\mathcal{T}_0 := \mathbb{F}_2$(第一个域也可以选择大一些的域$\mathbb{F}_{2^{2^k}}$,例如$\mathbb{F}_{16}$,最后计算结果都可以递归到第一个域$\mathbb{F}_{2^{2^k}}$中的计算,可以用查表的方法直接得到结果\cite{efficient-inversion-tower}),接着基于$\mathbb{F}_2$进行单代数扩张,考虑在$\mathbb{F}_2$上的不可约多项式$p(x) = x^2 + x + 1$,即$p(x)$在$\mathbb{F}_2$上找不到一个数$x_0$使得$p(x_0)=x_0^2+x_0+1=0$,但是在需要扩的域上$\mathbb{F}_{2^2}$上能够找到一个根,记这个根为$X_0 \in \mathbb{F}_{2^2}$,即$X_0^2+X_0+1 = 0$在$\mathbb{F}_{2^2}$上成立.设$\mathcal{T}_1 := \mathbb{F}_2[X_0]/(X_0^2+X_0+1)$为$\mathbb{F}_2$的扩域.考虑到能够进行递归的域扩张,对于$\forall \iota > 1$,
\begin{equation}\label{eq-extension-field}
    \mathcal{T}_{\iota} := \mathcal{T}_{\iota - 1}[X_{\iota - 1}]/ (X_{\iota - 1}^2 + X_{\iota - 2} \cdot X_{\iota - 1} + 1), \quad \quad X_{\iota - 2} \in \mathcal{T}_{\iota - 1}, X_{\iota - 1} \in \mathcal{T}_{\iota}.
\end{equation}
对于多项式$X_{\iota - 1}^2 + X_{\iota - 2} \cdot X_{\iota - 1} + 1$,将$X_{\iota - 1}$看作自变量$X$,多项式$X^2 + X_{\iota - 2} \cdot X + 1$在$\mathcal{T}_{\iota - 1}[X_{\iota - 1}]$上是不可约的(见\cite{iterated-quadratic}Thm. 1).下面写一些具体的例子来理解这个域扩张的过程。从$\mathbb{F}_2$开始:
\begin{displaymath}
    \mathcal{T}_0 := \mathbb{F}_2 = \{0,1\}
\end{displaymath}
对$\mathbb{F}_2$进行扩张,引入新的元素$X_0$,$\mathcal{T}_1 := \mathbb{F}_2[X_0]/(X_0^2+X_0+1)$,$\mathcal{T}_1$的$\mathcal{T}_0$-基为$1, X_0$,根据扩域的线性空间角度理解,$\mathcal{T}_1$中的元素都能写成$a + b X_0 (a,b \in \mathbb{T}_{\iota - 1})$的形式,则
\begin{displaymath}
    \mathcal{T}_1 := \mathbb{F}_2[X_0]/(X_0^2+X_0+1) = \{0,1, X_0, 1 + X_0\}.
\end{displaymath}
实际$\mathcal{T}_1 \cong \mathbb{F}_{2^2}$.接着扩域在$\mathcal{T}_1$上扩域
\begin{displaymath}
    \mathcal{T}_2 := \mathcal{T}_1[X_1]/(X_1^2+X_0X_1+1) 
\end{displaymath}
要计算$\mathcal{T}_2$中的元素,可以用一个表格来计算,见表\ref{table-F4},因为$\mathcal{T}_2$中的元素都可以用$\mathcal{T}_2$的$\mathcal{T}_2$-基为$\{1, X_1\}$线性表出,即写成$a + bX_1, a, b \in \mathcal{T}_1$.
\begin{table}[!htbp]
    \centering
    \caption{$\mathcal{T}_2$中的元素}\label{table-F4}
    \begin{tabular}{|c|c|c|c|c|}
    \hline
      $a / bX_1$ &  0 & $1 \cdot X_1$ & $X_0 \cdot X_1$ & $(1 + X_0) \cdot X_1$ \\
     \hline
    $0$ & $0$ & $X_1$ & $X_0X_1$ & $X_1+X_0X_1$\\
    \hline
    $1$ & $1$ & $1+X_1$ & $1+X_0X_1$ & $1+X_1+X_0X_1$\\
    \hline 
    $X_0$ & $X_0$ & $X_0+X_1$ & $X_0+X_0X_1$ & $X_0+X_1+X_0X_1$\\
    \hline 
    $1+X_0$ & $1+X_0$ & $1+X_0+X_1$ & $1+X_0+X_0X_1$ & $1+X_0+X_1+X_0X_1$\\
    \hline 
  \end{tabular}
\end{table}
继续扩域的过程是类似的.那么最后形成一个扩张塔$\mathcal{T}_0 \subset \mathcal{T}_1 \subset \cdots \mathcal{T}_{\iota}$.那么最后得到一个相同的环,$\forall \iota \ge 0$,
\begin{displaymath}
    \mathcal{T}_{\iota} = \mathbb{F}_2[X_0, \cdots, X_{\iota - 1}]/(X_0^2 + X_0 + 1, \cdots, X_{\iota - 1}^2 + X_{\iota - 2} \cdot X_{\iota - 1} + 1).
\end{displaymath}
这里的表示意思是可以从$\mathbb{F}_2$直接扩到$\mathcal{T}_{\iota}$,且$\mathbb{F}_2(X_0, \ldots, X_{\iota - 1}) \cong \mathbb{F}_2[X_0, \cdots, X_{\iota - 1}]/(X_0^2 + X_0+1,\ldots, X_{\iota - 1}^2 + X_{\iota - 2} \cdot X_{\iota - 1} + 1)$.
\begin{center}
    \fcolorbox{black}{gray!10}{\parbox{.9\linewidth}{\textbf{扩域中一次性添加多个元素}:在对域$F$扩张时,可以向域$F$中陆续添加多个元素$\alpha$,$\beta$,$\cdots$,$\gamma$,得到扩域$F(\alpha,\beta,\cdots,\gamma)$.即先向$F$添加$\alpha$,再向$F(\alpha)$添加$\beta$,等等.易知$F(\alpha,\beta,\cdots,\gamma)$即是$\alpha$,$\beta$,$\cdots$,$\gamma$和$F$的元素多次加减乘除得到的结果集合,是含$F$和$\alpha$,$\beta$,$\cdots$,$\gamma$的最小域.例如
    \begin{displaymath}
        \mathbb{Q}(\sqrt{2},\sqrt{3})= (\mathbb{Q}(\sqrt{2}))(\sqrt{3})=\{a+b\sqrt{2}+c\sqrt{3}+d\sqrt{6}|a,b,c,d \in \mathbb{Q}\}.
    \end{displaymath}}}
\end{center}
那就可以这样来理解$\mathbb{F}_2(X_0, \ldots, X_{\iota - 1}) \cong \mathbb{F}_2[X_0, \cdots, X_{\iota - 1}]/(X_0^2 + X_0+1,\ldots, X_{\iota - 1}^2 + X_{\iota - 2} \cdot X_{\iota - 1} + 1)$,是对$\mathbb{F}_2$一次性添加了$\iota$个元素$X_0, \ldots, X_{\iota - 1}$,也可以看成逐个去添加元素.扩张之后形成的域与上述的商环是同构的.

现在考虑在$\mathcal{T}_{\iota}$中的两个元素$a$和$b$做乘法运算,那么$a$与$b$都能用$\mathcal{T}_{\iota}$的$\mathcal{T}_{\iota - 1}$-基线性表示,由域扩张构造方程\eqref{eq-extension-field} $\mathcal{T}_{\iota} := \mathcal{T}_{\iota - 1}[X_{\iota - 1}]/ (X_{\iota - 1}^2 + X_{\iota - 2} \cdot X_{\iota - 1} + 1)$知$\mathcal{T}_{\iota} / \mathcal{T}_{\iota - 1}$的基为$\{1, X_{\iota - 1}\}$,则
\begin{displaymath}
    \begin{aligned}
        & a = a_0 + a_1 X_{\iota - 1}, \quad a_0,a_1 \in \mathcal{T}_{\iota - 1} \\
        & b = b_0 + b_1 X_{\iota - 1}, \quad b_0,b_1 \in \mathcal{T}_{\iota - 1}
    \end{aligned}
\end{displaymath}
因此$a$与$b$相乘可以写为:
\begin{displaymath}
    \begin{aligned}
        & (a_0 + a_1 X_{\iota - 1})(b_0 + b_1 X_{\iota - 1}) \\
        = &  a_0 b_0 + (a_0b_1+a_1b_0)X_{\iota - 1} + a_1b_1X_{\iota - 1}^2 \\
        & {\color{red} \text{(由于在}\mathcal{T}_{\iota}\text{中}X_{\iota - 1} \text{为不可约多项式}X_{\iota - 1}^2 + X_{\iota - 2} \cdot X_{\iota - 1}+1 \text{的根,因此}X_{\iota - 1}^2 + X_{\iota - 2}\cdot X_{\iota - 1}+1=0}\\
        & {\color{red}\text{,则}X_{\iota - 1}^2 = -X_{\iota - 2}\cdot X_{\iota - 1}-1.\text{在$\mathbb{F}_{2^k}$中,特征为$2$,则$1+1=0$,因此$X_{\iota - 1}^2 = X_{\iota - 2}\cdot X_{\iota - 1}+1$)}}\\
        = & a_0 b_0 + a_1b_1 + (a_0b_1+a_1b_0+a_1b_1X_{\iota - 2})X_{\iota - 1} 
    \end{aligned}
\end{displaymath}
这样在$\mathcal{T}_{\iota}$中比较大的两个数就能有效的转换为在比较小的$\mathcal{T}_{\iota - 1}$中的数相乘,这个过程利用了 Karatsuba 算法\cite{Karatsuba-technique},能够比直接在$\mathcal{T}_{\iota}$中进行计算更有效.利用算法复杂度分析中的主定理,可以得到这里的乘法复杂度为$\mathcal{O}(2^{\log3 \cdot \iota})\cite{efficient-inversion-tower}$.











\begin{thebibliography}{10}
    \bibitem{binius}Benjamin E. Diamond and Jim Posen, \href{https://eprint.iacr.org/2023/1784}{\textit{Succinct Arguments over Towers of Binary Fields}}, Cryptoglogy ePrint Archive, 2023.
    \bibitem{abstract-algerb}张贤科, \href{https://book.douban.com/subject/36007684/}{抽象代数}, 2022. 
    \bibitem{efficient-inversion-tower}John L. Fan and Christof Paar, \textit{On efficient inversion in tower fields of characteristic two}, Proceddings of IEEE International Symposium on Information Theory, 1997.
    \bibitem{vector-space}\href{https://zh.wikipedia.org/zh-cn/%E5%90%91%E9%87%8F%E7%A9%BA%E9%97%B4}{向量空间}, Wikipedia.
    \bibitem{iterated-quadratic}Doug Wiedemann, An Iterated Quadratic Extension of GF(2), The Fibonacci Quarterly, 1988.
    \bibitem{Karatsuba-technique}\href{https://en.wikipedia.org/wiki/Karatsuba_algorithm}{Karatsuba Algorithm}, Wikipedia.
\end{thebibliography}
\end{document}